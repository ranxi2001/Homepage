% !TEX TS-program = xelatex
% !TEX encoding = UTF-8 Unicode
% !Mode:: "TeX:UTF-8"

\documentclass{resume}
\usepackage{zh_CN-Adobefonts_external} % Simplified Chinese Support using external fonts (./fonts/zh_CN-Adobe/)
%\usepackage{zh_CN-Adobefonts_internal} % Simplified Chinese Support using system fonts
\usepackage{linespacing_fix} % disable extra space before next section
\usepackage{cite}
\usepackage[hidelinks]{hyperref}

\begin{document}
\pagenumbering{gobble} % suppress displaying page number

\name{冉 熙}

\contactInfo{\href{mailto:ranxi@88.com}{ranxi@88.com}}{\href{https://ranxi.top}{ranxi.top}}{191xxxxx913}{\href{https://github.com/ranxi2001}{GitHub @ranxi2001}}
 
\section{个人信息}
\begin{itemize}[parsep=0.2ex]
  \item \textbf{求职意向}: Web3产品研发 | 量化交易与风控研究员 | 智能合约开发工程师
  \item \textbf{基本信息}: 2001.09出生 | 籍贯:湖北利川 | 现居住地:浙江宁波| 研二在读
\end{itemize}

\section{教育背景}
\datedsubsection{\textbf{中国科学院大学}}{2024.08 -- 2027.07}
\textit{硕士}, 计算机技术 (杭州高等研究院 \, 智能科学与技术学院 \, \textit{杭州})
\begin{itemize}
  \item \textbf{学业成绩}: GPA: 3.75 \hspace{4em} \textbf{研究领域}: 区块链、Web3、量化与风控
  \item \textbf{核心课程}: 区块链导论, 计算机体系结构, 并行计算, 自然语言处理, 高级人工智能, 机器人与智能计算
\end{itemize}

\datedsubsection{\textbf{浙江工商大学}}{2020.09 -- 2024.07}
\textit{学士}, 信息管理与信息系统 (管理工程与电子商务学院 \, \textit{杭州})
\begin{itemize}
  \item \textbf{学业成绩}: GPA: 3.77 \hspace{4em} \textbf{研究领域}: 区块链、NLP、数学建模
  \item \textbf{核心课程}: 网络程序设计(91), 数据统计与可视化(96), 大数据关键技术与应用(94), 机器学习(89), Android移动应用开发(95)
  \item \textbf{英语水平}: CET-4: 535 | CET-6: 437 \, (具备良好的英语科技论文阅读写作能力)
\end{itemize}

\section{专业技能}
\begin{itemize}[parsep=0.2ex]
  \item \textbf{编程语言}: 熟悉 Python, Java, Solidity
  \item \textbf{框架与工具}: Pytorch, PaddlePaddle, TensorFlow, React, Android UI, Git, Docker, LaTeX
  \item \textbf{算法能力}: 熟悉机器学习、深度学习和启发式优化算法,具备数据分析与算法设计能力
\end{itemize}

\section{工作与实习经历}
\datedsubsection{\textbf{浙江大学}, 加密量化风控算法研发}{2025.10 - 至今}
\begin{itemize}
  \item 开发基于Agent的多变量融合风控算法系统,整合量价、情绪和链上指标。
  \item 构建Python算法风控平台,包含账户风险仪表盘和策略管理接口。
\end{itemize}

\datedsubsection{\textbf{浙江大学}, 加密货币量化研究}{2025.06 - 2025.10}
\begin{itemize}
  \item 开发基于推送信号的Solana链上主要交易框架,实现长期可持续盈利。
  \item 构建基于LLM Agent的二级交易所量化新闻交易框架,实现85\%胜率和持续盈利。
\end{itemize}

\datedsubsection{\textbf{杭州趣链科技有限公司}, 区块链研发工程师实习}{2023.10 - 2024.06}
\begin{itemize}
  \item 区块链全栈学习;撰写科技报告;行业投研;Hyperchain部署应用开发与测试。
\end{itemize}

\datedsubsection{\textbf{图灵学院}, 研究助理}{2023.06 - 2023.09}
\begin{itemize}
  \item 协助NLP研究项目;应用Python与GPT-4/Claude 2进行研究任务;参与模型评估与优化。
\end{itemize}

\datedsubsection{\textbf{浙江有数数字科技有限公司}, 数据分析实习生}{2022.01 - 2022.02}
\begin{itemize}
  \item 协助政务大数据分析;使用Excel, Python等工具进行数据处理和可视化。
\end{itemize}

\datedsubsection{\textbf{自媒体运营 (博客/知乎/公众号)}, 主理人}{2020.11 - 至今}
\begin{itemize}
  \item 撰写技术文章,公众号粉丝2w+,知乎阅读125w+,博客园阅读20w+。
\end{itemize}

\section{科研与竞赛经历}
\datedsubsection{\textbf{挑战杯揭榜挂帅项目“融智图谱”} (全国二等奖), \textit{负责人}}{2023.04 - 2023.09}
\begin{itemize}
  \item 基于飞桨UIE模型微调(Finetune)多模态文本,实现96\%以上F1-Score。
  \item 搭建Doccano平台实现协作标注;封装Flask API实现自动标注与人工校对。
\end{itemize}

\datedsubsection{\textbf{浙江省新苗人才计划“越岚数聚”}, \textit{负责人}}{2021.08 - 2023.06}
\begin{itemize}
  \item 利用区块链技术构建数据与课程共享体系。授权4项软著,受理2项发明专利。
\end{itemize}

\datedsubsection{\textbf{国家发明专利“基于区块链和IPFS的图像文件共享平台侵权保护方法”}, \textit{第一发明人}}{2021.12}
\begin{itemize}
  \item 采用感知哈希(pHash)算法结合区块链与IPFS,在无人工审核下有效保护创作者知识产权。
\end{itemize}

\section{发表论文}
\begin{itemize}[parsep=0.2ex]
  \item Sun, S.; \textbf{Ran, X.}; Pang, S.; Chen, X.*; Sun, Y.* "DataR2E: Research and Prospects on the Value Release of Data Elements in Web 3.0". \textit{IEEE Global Blockchain Conference (GBC)}, 2025.
  \item Ju, C.; Shen, Z.; Bao, F.*; Wen, Z.; \textbf{Ran, X.}; Yu, C.; Xu, C. "Blockchain Traceability System in Complex Application Scenarios: Image‐Based Interactive Traceability Structure". \textit{Systems}, 2022, 10, 78. (JCR Q2)
\end{itemize}

\section{荣誉奖项}
\begin{tabular}{@{}l@{\hspace{1em}}p{0.42\textwidth}p{0.42\textwidth}}
\textbf{国家级}: & 2024研究生数模竞赛二等奖 (1/3) & 2022 美国大学生数学建模 ICM H奖 (1/3) \\
                 & 2023挑战杯揭榜挂帅全国二等奖 (1/10) & 2021国家励志奖学金 \\[0.5ex]
\textbf{省部级}: & 2022浙江省政府奖学金 (前5\%) & 2022全国大学生数模竞赛省二等奖 (1/3) \\
                 & 2022省挑战杯铜奖 (1/10) & 2022省电子商务竞赛二等奖 (3/5) \\
                 & 2022省物流设计竞赛二等奖 (4/5) & 2021省互联网+铜奖 \\[0.5ex]
\textbf{校级}:   & 2025国科大学业二等奖学金 (前20\%) & 2021、2022综合一等奖学金 (前3\%) \\
                 & 2022海平专项奖学金 (全校仅30名) & 2022勤工助学之星 (全校仅10名) \\
                 & 2023综合能力突出奖学金 & 2022三好学生 (前5\%)
\end{tabular}

\section{社区参与}
\begin{itemize}[parsep=0.2ex]
  \item \textbf{中国科学院大学区块链协会}: 核心成员 (2024.09 - 至今)
  \item \textbf{浙江大学区块链协会}: 核心成员 (2023.10 - 至今)
\end{itemize}

\end{document}
